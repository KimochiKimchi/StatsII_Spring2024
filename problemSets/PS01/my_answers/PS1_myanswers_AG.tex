\documentclass{article}
\usepackage{graphicx} % Required for inserting images

\title{PS1_myanswers_AG}
\author{Aryan Goyal}
\date{February 2024}

\begin{document}

\section{PS1 Answers}
Aryan Goyal

\noindent Student number: 18306046
\vspace{0.2cm}

\section{Question 1}
Write an R function that implements the Kolmogorov-Smirnov test where the reference distribution is normal. Using R generate 1,000 Cauchy random variables (rcauchy(1000, location = 0, scale = 1)) and perform the
test (remember, use the same seed, something like set.seed(123), whenever you’re generating your own data).

\subsection{Answer 1}
First I set the seed for reproducibility:
\begin{verbatim}
set.seed(100)
empirical <- rcauchy(1000, location = 0, scale = 1)
\end{verbatim}
\\
Next, I write a function to manually conduct the Kolmogorov-Smirnov test in R:
\begin{verbatim}    
KS_function <- function (data){
  ECDF <- ecdf(data)
  empiricalCDF <- ECDF(data)
  
# generate test statistic
  D <- max(abs(empiricalCDF - pnorm(data)))
  
# Calculating the p-value
  i_values <- 1:1000
  sum1 <- sum(exp(-((2 * i_values - 1)^2 * pi^2) / ((8 * D)^2)))
  p_value <- sqrt(2 * pi) / D * sum1
  
# Return result
  result <- list(
  Test_statistic = D,
  P_value = p_value
)
  return(result)
}

# Performing the K-S test with our data
KS_function(empirical)
    
\end{verbatim}
\\
The function returns a list containing the test statistic ('D') and the p-value. The test statistic is 0.13997 and the p-value is 0.00683
    Hence, the p-value is below the 0.05 level of significance
\\
\vspace{0.1cm}
I compare the KS\_function created by me, with the in-built function in R to check the results:
\\
\begin{verbatim}
ks.test(empirical,"pnorm")
\end{verbatim}
\\
In this case, the test statistic is 0.14097 which is very close to the fuction and the p-value is 2.2e-16 
\vspace{0.1cm}
\\
When comparing the in-built function with the by-hand function, the results we get are quite similar. Both the p-values are below the 0.05 level of significance and the test statistics are also very similar (0.13997 for by-hand and 0.14097 for the in-built function)

\section{Question 2}

Estimate an OLS regression in R that uses the Newton-Raphson algorithm (specifically BFGS, which is a quasi-Newton method), and show that you get the equivalent results to using lm.

\subsection{Answer 2}
\begin{verbatim}
# Set seed for reproducibility and using the provided code to create the 
# data
set.seed(191)
data <- data.frame(x = runif(200, 1, 10))
data$y <- 0 + 2.75 * data$x + rnorm(200, 0, 1.5)
\end{verbatim}
\\
Next, I initiate a function to calculate the log-likelihood for OLS
\begin{verbatim}
#Log-likelihood function for OLS (taking help from Tutorial 2 script)
log_likelihood <- function(outcome, input, parameter) {
  y_hat <- beta[1] + beta[2] * x
  residuals <- y - y_hat
  log_likelihood1 <- 0.5 * sum((residuals / 1.5)^2) + 0.5 * length(residuals) * log(2 * pi * 1.5^2)
  return(log_likelihood1)
}

\end{verbatim}
I set up the initial guess for coefficients for iteration and then optimize them using the BFGS algorithm through the optim() function
\begin{verbatim}    
# Initial guess for coefficients for iteration
initial_guess <- c(0, 0)

# Optimize using BFGS
?optim
result_bfgs <- optim(par = initial_guess,fn = log_likelihood, 
                     x = data$x, y = data$y, method = "BFGS")

\end{verbatim}
Finally, I extract the coefficients from my function and compare them to the in-built lm function in R:
\begin{verbatim}    
# Extract coefficients from BFGS result
coefficients_bfgs <- result_bfgs$par
print(coefficients_bfgs)

# OLS Regression using lm
ols_lm <- lm(y ~ x, data = data)

# Extract coefficients from lm result
print(coef(ols_lm))
\end{verbatim}
\\
Using the BFGS method, the intercept is 0.15734 and $\beta_1$ is 2.76752

    Whereas, the intercept is 0.15208 and $\beta_1$ is 2.76760 using the in-built lm() function in R.
\\
Hence, the by-hand method gets close to the in-built function in R

\end{document}
