\documentclass{article}
\usepackage{graphicx}
\usepackage{float}

\title{Problem Set 3}
\author{Aryan Goyal}
\date{March 2024}

\begin{document}

\maketitle

First, I load in the data and inspect it:

\begin{verbatim}
gdp_data <- read.csv("https://raw.githubusercontent.com/ASDS-
TCD/StatsII_Spring2024/main/datasets/gdpChange.csv", stringsAsFactors = F)

head(gdp_data)
\end{verbatim}

\section{Question 1}
\subsection{Construct and interpret an unordered multinomial logit with GDPWdiff as the output
and ”no change” as the reference category, including the estimated cutoff 
points and coefficients.}
\begin{verbatim}
    # First, I convert the GDP to a factor variable
gdp_data[gdp_data$GDPWdiff==0, "factor_GDP"] <- "no change"
gdp_data[gdp_data$GDPWdiff>0, "factor_GDP"] <- "positive"
gdp_data[gdp_data$GDPWdiff<0, "factor_GDP"] <- "negative"
gdp_data$factor_GDP <- relevel(as.factor(gdp_data$factor_GDP), ref="no change")
\end{verbatim}
I run an unordered multinomial logit with releveled GDP variable as the output variable and "no change" as the reference category
\begin{verbatim}
unorderedLogit <- multinom(factor_GDP ~ REG + OIL, data=gdp_data)
summary(unorderedLogit)
\end{verbatim}
These are the results:
\\
\begin{table} [H]
\begin{center}
\begin{tabular}{l c}
\hline
 & Model 1 \\
\hline
negative: (Intercept) & $3.81^{***}$ \\
                      & $(0.27)$     \\
negative: REG         & $1.38$       \\
                      & $(0.77)$     \\
negative: OIL         & $4.78$       \\
                      & $(6.89)$     \\
positive: (Intercept) & $4.53^{***}$ \\
                      & $(0.27)$     \\
positive: REG         & $1.77^{*}$   \\
                      & $(0.77)$     \\
positive: OIL         & $4.58$       \\
                      & $(6.89)$     \\
\hline
AIC                   & $4690.77$    \\
BIC                   & $4728.10$    \\
Log Likelihood        & $-2339.39$   \\
Deviance              & $4678.77$    \\
Num. obs.             & $3721$       \\
K                     & $3$          \\
\hline
\multicolumn{2}{l}{\scriptsize{$^{***}p<0.001$; $^{**}p<0.01$; $^{*}p<0.05$}}
\end{tabular}
\caption{Unordered Multinomial Logit Model}
\label{table:coefficients}
\end{center}
\end{table}

Based on the results, I can make the following interpretations.
\\
The intercepts:
\\
Keeping in mind that the reference category is "no change", the "negative" intercept tells us that when REG and OIL are both = 0, the estimated log odds of going from no change to negative is 3.8. 
\\
Similarly, the "positive" intercept tells us that when REG and OIL are both = 0, the estimated log odds of going from no change to positive is 4.53. 
\\
For REG:
\\
A 1 unit increase in REG (going from a non-democracy to a democracy) corresponds with a change in the log odds of going from no change to negative by 1.38, while holding OIL constant. 
\\
A 1 unit increase in REG (going from a non-democracy to a democracy) corresponds with a change in the log odds of going from no change to positive by 1.77, while holding OIL constant.
\\
For OIL:
\\
A 1 unit increase in OIL corresponds with a change in the log odds of going from no change to negative by 4.78, while holding REG constant. 
\\
A 1 unit increase in OIL corresponds with a change in the log odds of going from no change to positive by 4.58, while holding REG constant.

\subsection{Construct and interpret an ordered multinomial logit with GDPWdiff as the outcome
variable, including the estimated cutoff points and coefficients.}

First, I relevel the variable to create an ordinal structure
\begin{verbatim}
gdp_data$factor_GDP1 <- relevel(gdp_data$factor_GDP, ref="negative")
\end{verbatim}

Next, I use the polr() function to conduct an ordered multinomial logit regression
\begin{verbatim}
OrderedModel <- polr(factor_GDP1 ~ REG + OIL, data=gdp_data)
summary(OrderedModel)
\end{verbatim}

Here are the results:
\begin{table} [H]
\begin{center}
\begin{tabular}{l c}
\hline
 & Model 1 \\
\hline
REG                & $0.40^{***}$  \\
                   & $(0.08)$      \\
OIL                & $-0.20$       \\
                   & $(0.12)$      \\
negative|no change & $-0.73^{***}$ \\
                   & $(0.05)$      \\
no change|positive & $-0.71^{***}$ \\ 
                   & $(0.05)$      \\
\hline
AIC                & $4695.69$     \\
BIC                & $4720.58$     \\
Log Likelihood     & $-2343.84$    \\
Deviance           & $4687.69$     \\
Num. obs.          & $3721$        \\
\hline
\multicolumn{2}{l}{\scriptsize{$^{***}p<0.001$; $^{**}p<0.01$; $^{*}p<0.05$}}
\end{tabular}
\caption{Ordered Multinomial Logit Model}
\label{table:coefficients}
\end{center}
\end{table}

Next, I interpret the ordered multinomial logit model.
\\
For REG:
\\
A 1 unit increase in REG (going from non-democracy to democracy) corresponds to a change in the log odds of going from negative to no change and from no change to positive by 0.40, while holding OIL constant. 
\\
For OIL:
\\
A 1 unit increase in OIL corresponds to a change in the log odds of going from negative to no change and from no change to positive by -0.20, while holding REG constant. 


\section{Question 2}
\subsection{Run a Poisson regression because the outcome is a count
variable. Is there evidence that PAN presidential candidates visit swing 
districts more? Provide a test statistic and p-value)}

I run a Poisson regression using the glm() function:
\begin{verbatim}
    poissonModel <- glm(PAN.visits.06 ~ competitive.district + marginality.06 + 
    PAN.governor.06, data = mexico_elections, family=poisson)
summary(poissonModel)

\end{verbatim}

Here are the results:
\begin{table} [H]
\begin{center}
\begin{tabular}{l c}
\hline
 & Model 1 \\
\hline
(Intercept)          & $-3.81^{***}$ \\
                     & $(0.22)$      \\
competitive.district & $-0.08$       \\
                     & $(0.17)$      \\
marginality.06       & $-2.08^{***}$ \\
                     & $(0.12)$      \\
PAN.governor.06      & $-0.31$       \\
                     & $(0.17)$      \\
\hline
AIC                  & $1299.21$     \\
BIC                  & $1322.36$     \\
Log Likelihood       & $-645.61$     \\
Deviance             & $991.25$      \\
Num. obs.            & $2407$        \\
\hline
\multicolumn{2}{l}{\scriptsize{$^{***}p<0.001$; $^{**}p<0.01$; $^{*}p<0.05$}}
\end{tabular}
\caption{Poisson model}
\label{table:coefficients}
\end{center}
\end{table}

To answer this question, I interpret the competitive.district coefficient.
A 1 unit increase in competitive.district (going from "safe seat" to "swing district") has a multiplicative effect on the mean of Poisson by 
$e^-^0^.^0^8$, while holding all other variables constant. 
\\
exp(-0.08) = 1.083287. 
\\
As $B_1 > 0$, the expected count increases as number of visits increases.
However, the test statistic (-0.477) is not large enough and this coefficient is not significant at the critical threshold($p-value = 0.6336 > 0.05$) and hence, we do not have enough evidence to suggest that PAN presidential candidates visit swing districts more.

\subsection{Interpret the marginality.06 and PAN.governor.06 
coefficients)}

For marginality.06: 
\\
$exp(-2.08) = 0.13$
\\
A 1 unit increase in poverty corresponds with a decrease in the expected number of visits by a multiplicative factor of 0.13, while holding PAN.governor.06 and competitive.district constant.
\\
\\
For PAN.governor.06:
\\
$exp(-0.31) = 0.73$
\\
In comparison to not having a PAN-affiliated governor, having a PAN-affiliated governor corresponds to a decrease in the expected number of visits by a multiplicative factor of 0.73, while holding marginality.06 and competitive.district constant.

\subsection{Provide the estimated mean number of visits from the winning 
PAN presidential candidate for a hypothetical district that was 
competitive (competitive.district=1), had an average poverty level 
(marginality.06 = 0), and a PAN governor (PAN.governor.06=1).}

I make use of the predict() function:
\begin{verbatim}
    predict(poissonModel, newdata=data.frame(competitive.district=1, 
                                    marginality.06 = 0, 
                                    PAN.governor.06=1), type="response")

\end{verbatim}

Based on this, I get the value: 0.01494818 
\\
When a district is competitive, has an average poverty level and has a PAN-affiliated governor, the expected number of visits based on our model is 0.01.


\end{document}
