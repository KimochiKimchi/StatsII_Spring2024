\documentclass{article}
\usepackage{graphicx}
\usepackage{float}
\usepackage{listings}

\title{Problem Set 4}
\author{Aryan Goyal}
\date{April 2024}

\begin{document}

\maketitle

First, I load in the data and inspect it:

\begin{lstlisting}[language=R]
data("child")
head(child)
summary(child)    
\end{lstlisting}
    
\section{Question 1}

We’re interested in modeling the historical causes of child mortality. We 
have data from 26855 children born in Skelleftea, Sweden from 1850 to 
1884. Using the ”child” dataset in the eha library, fit a Cox 
Proportional Hazard model using mother’s age and infant’s gender
as covariates. Present and interpret the output.

\section{Answer}

\begin{lstlisting}

child_surv <- with(child, Surv(enter, exit, event))

Model <- coxph(child_surv ~ m.age + sex, data = child)
summary(Model)

#Getting code for LaTex
stargazer(Model)
\end{lstlisting}



\begin{table}[H] \centering 
  \caption{Cox Proportional Hazard Model} 
  \label{} 
\begin{tabular}{@{\extracolsep{5pt}}lc} 
\\[-1.8ex]\hline 
\hline \\[-1.8ex] 
 & \multicolumn{1}{c}{\textit{Dependent variable:}} \\ 
\cline{2-2} 
\\[-1.8ex] & child\_surv \\ 
\hline \\[-1.8ex] 
 m.age & 0.008$^{***}$ \\ 
  & (0.002) \\ 
  & \\ 
 sexfemale & $-$0.082$^{***}$ \\ 
  & (0.027) \\ 
  & \\ 
\hline \\[-1.8ex] 
Observations & 26,574 \\ 
R$^{2}$ & 0.001 \\ 
Max. Possible R$^{2}$ & 0.986 \\ 
Log Likelihood & $-$56,503.480 \\ 
Wald Test & 22.520$^{***}$ (df = 2) \\ 
LR Test & 22.518$^{***}$ (df = 2) \\ 
Score (Logrank) Test & 22.530$^{***}$ (df = 2) \\ 
\hline 
\hline \\[-1.8ex] 
\textit{Note:}  & \multicolumn{1}{r}{$^{*}$p$<$0.1; $^{**}$p$<$0.05; $^{***}$p$<$0.01} \\ 
\end{tabular} 
\end{table}


I fit a Cox  Proportional Hazard model using mother’s age and infant’s gender as covariates. The table above show the result of my model.
\\
\\
Interpretation of estimated coefficient for age:
\\
The logged hazard ratio converts to a hazard ratio of exp(0.008) = 1.008.
Holding infant gender constant, a one unit increase in mother's age is associated with an increase in the logged hazard ratio for infants of the same gender.This means that for every one-year increase in mother's age, the hazard of death for the child increases by a multiplicative factor of 1.008. Hence, as mothers get older, the risk of child mortality also increases on average.
\\
\\
Interpretation of infant's gender:
\\
The negative coefficient for "sexfemale" translates to a hazard ratio of exp(-0.08) = 0.92. 
Holding mother's age constant, the infant's sex being female is associated with a 0.92 times lower hazard of death compared to males on average. 
\end{document}
